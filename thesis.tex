
\documentclass[12pt]{article}     
\usepackage{graphicx}
\graphicspath{{figures/}}
\usepackage[top=2.5cm, bottom=2.5cm, left=3cm, right=3cm]{geometry}
\usepackage{titlesec}
\usepackage{longtable}
\usepackage[table,xcdraw]{xcolor}
\usepackage{todonotes}
\usepackage{float}
\usepackage[T1]{fontenc}
\usepackage[utf8]{inputenc}
\usepackage{csquotes}
\usepackage{tocloft}
\usepackage{amssymb}
\renewcommand{\labelitemi}{\tiny$\blacksquare$} %For square itemized lists
\usepackage{caption} 
\captionsetup{labelsep=period}
\usepackage{verbatimbox} %To put program code in the center using Verbatim
\titlelabel{\thetitle.\quad}
\usepackage{times}
\usepackage{fancyhdr}
\setlength{\parindent}{0cm}
\usepackage{setspace}
\onehalfspacing
\setlength{\parskip}{\baselineskip}
\usepackage{amsmath} 
\usepackage{amsthm}
% Packages for building tables and tabulars 
\usepackage{array}
\usepackage{tabu}   % Wide lines in tables
\usepackage{xspace} % Non-eatable spaces in macros
\usepackage[colorlinks=true,linkcolor=blue]{hyperref}
\usepackage[all]{hypcap}
\usepackage{url}
% Packages for defining colourful text together with some colours
%\usepackage[table,xcdraw]{xcolor}
\definecolor{dkgreen}{rgb}{0,0.6,0}
\definecolor{gray}{rgb}{0.5,0.5,0.5}
\definecolor{mauve}{rgb}{0.58,0,0.82}
\definecolor{lightblue}{rgb}{0.95,0.97,1.0}
\definecolor{darkblue}{rgb}{0.90,0.92,1.0}
\usepackage{color}

% Standard package for drawing algorithms
% Since the thesis in article format we must define \chapter for
% the package algorithm2e (otherwise obscure errors occur) 
\let\chapter\section
\usepackage{algorithm2e}

% Macros that make sure that the math mode is set
\newcommand{\typeF}[1] {\ensuremath{\mathsf{type_{#1}}}\xspace}
\newcommand{\opDiv}{\ensuremath{\backslash \mathsf{div}}\xspace} 
\usepackage{listings}

\lstset{ 
  %language=python,                % the language of the code
  language=C++,
  basicstyle=\footnotesize,        % the size of the fonts that are used for the code
  %numbers=left,                   % where to put the line-numbers
  %numberstyle=\footnotesize,      % the size of the fonts that are used for the line-numbers
  numberstyle=\tiny\color{gray}, 
  stepnumber=1,                    % the step between two line-numbers. If it's 1, each line 
                                   % will be numbered
  numbersep=5pt,                   % how far the line-numbers are from the code
  backgroundcolor=\color{white},   % choose the background color. You must add \usepackage{color}
  showspaces=false,                % show spaces adding particular underscores
  showstringspaces=false,          % underline spaces within strings
  showtabs=false,                  % show tabs within strings adding particular underscores
  frame = lines,
  %frame=single,                   % adds a frame around the code
  rulecolor=\color{black},		   % if not set, the frame-color may be changed on line-breaks within 
                                   % not-black text (e.g. commens (green here))
  tabsize=2,                       % sets default tabsize to 2 spaces
  captionpos=b,                    % sets the caption-position to bottom
  breaklines=true,                 % sets automatic line breaking
  breakatwhitespace=false,         % sets if automatic breaks should only happen at whitespace
  %title=\lstname,                 % show the filename of files included with \lstinputlisting;
                                   % also try caption instead of title
                                   % also try caption instead of title
  keywordstyle=\color{blue},       % keyword style
  commentstyle=\color{dkgreen},    % comment style
  stringstyle=\color{mauve},       % string literal style
  escapeinside={\%*}{*)},          % if you want to add a comment within your code
  morekeywords={*,game, fun}       % if you want to add more keywords to the set
}

\usepackage{multirow}
\setlength{\tabcolsep}{0pt}
\newcolumntype{C}[1]{>{\centering\let\newline\\\arraybackslash\hspace{0pt}}m{#1-\arrayrulewidth\relax}}
\newcolumntype{L}[1]{>{\raggedright\let\newline\\\arraybackslash\hspace{0pt}}m{#1-\arrayrulewidth\relax}}

\usepackage{booktabs,fixltx2e}
\usepackage{tikz}
%used for ex. for m prime
\usepackage{flexisym}
\usepackage{footnote}

\usepackage{arydshln}
\usepackage{placeins}
\usepackage{enumitem}
\usepackage{cleveref}
\crefformat{footnote}{#2\footnotemark[#1]#3}
\usepackage{IEEEtrantools}

\begin{document}
\bstctlcite{IEEEexample:BSTcontrol}
%------------------------------TIITELLEHT---------------------------------
\thispagestyle{fancy}
\renewcommand{\headrulewidth}{0pt}
\renewcommand{\footrulewidth}{0pt}
\headheight = 57pt
\footskip = 11pt
\headsep = 0pt

\chead{
 \textsc{\begin{Large} %Tekst suurtähtedega ja suuremaks
	Tallinn University of Technology\\
	\end{Large} }
	Department of Computer Science\\	
	TUT Center for Digital Forensics and Cyber Security
}
\vspace*{7 cm}

\begin{center}
ITC70LT\\[0cm]

Gvantsa Grigolia 144965\\
\vspace{15pt}
\begin{LARGE}
\textsc{Evaluation of data ownership solutions in remote storage.\\}
\end{LARGE}
\vspace{10pt}
Master Thesis\\[2cm]
\end{center}

\begin{flushright}
Supervisor: Ahto Buldas\\[0cm]
Professor \\[0cm]
\end{flushright}

\cfoot{Tallinn 2016} 
\pagebreak

%---------------------------AUTORIDEKLARATSIOON-------------------------
\section*{\begin{center}
 Autorideklaratsioon
\end{center}}


Autorideklaratsioon on iga lõputöö kohustuslik osa, mis järgneb tiitellehele.
Autorideklaratsioon esitatakse järgmise tekstina:

Olen koostanud antud töö iseseisvalt. Kõik töö koostamisel kasutatud teiste autorite tööd, olulised seisukohad, kirjandusallikatest ja mujalt pärinevad andmed on viidatud. Käsolevat tööd ei ole varem esitatud kaitsmisele kusagil mujal.

Autor: [Ees$-$ ja perenimi]

[\today]
\pagebreak

%---------------------------ANNOTATION---------------------------------
\section*{\begin{center}
Annotatsioon
\end{center}}

Annotatsioon on lõputöö kohustuslik osa, mis annab lugejale ülevaate töö eesmärkidest, olulisematest käsitletud probleemidest ning tähtsamatest tulemustest ja järeldustest. Annotatsioon on töö lühitutvustus, mis ei selgita ega põhjenda midagi, küll aga kajastab piisavalt töö sisu. Inglisekeelset annotatsiooni nimetatakse Abstract, venekeelset aga


Sõltuvalt töö põhikeelest, esitatakse töös järgmised annotatsioonid:
\begin{itemize}
\item kui töö põhikeel on eesti keel, siis esitatakse annotatsioon eesti keeles mahuga $\frac{1}{2	}$ A4 lehekülge ja annotatsioon \textit{Abstract} inglise keeles mahuga vähemalt 1 A4 lehekülg;
\item kui töö põhikeel on inglise keel, siis esitatakse annotatsioon (Abstract)  inglise keeles mahuga $\frac{1}{2}$ A4 lehekülge ja annotatsioon eesti keeles mahuga vähemalt 1 A4 lehekülg;
\end{itemize}

Annotatsiooni viimane lõik on kohustuslik ja omab järgmist sõnastust:

Lõputöö on kirjutatud [mis keeles] keeles ning sisaldab teksti [lehekülgede arv] leheküljel, [peatükkide arv] peatükki, [jooniste arv] joonist, [tabelite arv] tabelit.
\pagebreak


%-----------------------------ABSTRACT-----------------------------------

\section*{\begin{center}
Abstract
\end{center}}
Võõrkeelse annotatsiooni koostamise ja vormistamise tingimused on esitatud eestikeelse annotatsiooni juures.

The thesis is in [language] and contains [pages] pages of text, [chapters] chapters, [figures] figures, [tables] tables.
\pagebreak

%---------------------Glossary of terms and Abbreviations---------------------

%\section*{\begin{center}
%Glossary of Terms and Abbreviations
%\end{center}}
%Lühendite  ning  mõistete  sõnastikku  lisatakse kõik töö põhitekstis kasutatud  uued  ning  ka mitmetähenduslikud üldtuntud terminid. Näiteks inglisekeelne lühend PC  võib tähendada nii Personal Computer kui ka Program Counter, sõltuvalt kontekstist. Lühendid ja mõisted esitatakse tabuleeritult kahte tulpa selliselt, et vasakul on esitatud lühend või mõiste ja paremal tulbas seletus. Inglisekeelsed sõnad seletustes esitatakse kaldkirjas. Alltoodud näited esitavad lühendite ja mõistete sõnastiku korrektset vormistamist.

%\begin{tabular}{p{3 cm}ll}
%IPv6&Internet Protocol version 6\\
%ICMPv6&Internet Control Message Protocol version 6\\
%Node&ll\\
%NAT&dd\\
%IANA&Internet Assigned Numbers Authority\\
%BYID&Bring Your Own Device\\
%OS&Operating System\\
%IoT&Internet of Things\\
%rootkit&ff
%\end{tabular}
%\pagebreak

\tableofcontents
\newpage
\listoffigures
\pagebreak
\listoftables
\pagebreak

\section{Introduction}
\label{sec:1}

Describes the problem statement, illustrates why this is a problem and describes the contribution the thesis makes in solving this problem. Optionally, 
it can give a short description (1-3 sentences each) of the remaining chapters. Good introductions are concise, typically no longer than 4 pages.\\
%The introduction reveals the full (but summarized) results of your work. This appears counter-intuitive: does this not break the tension, 
%like revealing the name of the murderer on the first page of a thriller? Yes, it does. That is the whole point. A thesis, and thus its architecture, 
%aims primarily to inform, not entertain.

\pagebreak
%---------------------------------TERMS AND DEFINITION -----------------------------------------
\section{Terms and definition}
\label{sec:2}
Defines the fundamental concepts your thesis builds on. Your thesis implements a new type of parser generator and uses the term non-terminal symbol a lot? 
Here is where you define what you mean by it. The key to this chapter is to keep it very, very short. 
Whenever you can, don’t reinvent a description for an established concept, but reference a text book or paper instead.
\pagebreak
%------------------------------BACKGROUND AND RELATED WORK-----------------------------------
\section{Background and Related Work}
\label{sec:3}


\subsection{Data Deduplication}
\label{sub:background}

The aim of data deduplication is to reduce the disk space and bandwidth needed to transfer the data. 



Describe how deduplication works. 


\subsection{Related Work}
\label{sub:covert}

Collects descriptions of existing work that is related to your work. Related, in this sense, means aims to solve the same problem or uses the same approach to solve a different problem. 

%This chapter typically reads like a structured list. Each list item summarizes a piece of work (typically a research paper) briefly and explains the relation to your work. 
%This last part is absolutely crucial: the reader should not have to figure out the relation himself.
%Is your piece better from some perspective? More generalizable? More performant? Simpler? It is ok if it is not, but I want you to tell me.

\subsection{Summary}

\pagebreak

%------------------------------APPROACH-----------------------------------
\section{Approach}
\label{sec:4}

%Outlines the main thing your thesis does. Your thesis describes a novel algorith for X? Your main contribution is a case study that replicates Y? Describe it here.

\begin{enumerate}

  \item Review existing solutions of data ownership problem in de-duplicated cloud storage. 

  \item Identified most favorable cases in terms of  security and efficiency and cost. 

  \item Based on those case  build a framework, which will help customers to choose secure and efficient approaches.     

\end{enumerate}



\pagebreak

%------------------------------EVALUATION-----------------------------------
\section{Evaluation}
\label{sec:5}
Demonstrate why the developed framework  is secure and efficient. 


\begin{enumerate}
 \item Time complexity evaluation.
 \item Cost analyses. 
\end{enumerate}



%Describes why your approach really solves the problem it claims to solve. You implemented a novel algorithm for X? 
%This chapter describes how you ran it on a dataset and reports the results you measured. You replicated a study? This chapter gives the results and your interpretations.

%The Appraoch and Evaluation chapters contain the meat of your thesis. Often, they make up half or more of the pages of the entire document.

\pagebreak

%------------------------------FUTURE WORK-----------------------------------
\section{Future Work}
\label{sec:6}

 In science folklore, the merit of a research question is compounded by the number of interesting follow-up research questions it raises. 
 So to show the merit of the problem you worked on, you list these questions here.
 %If you don’t care about research folklore (I did not as a student),
 %this chapter is still useful: whenever you stumble across something that you should do if you had unlimited time, but cannot do since you don’t, you describe it here. 
 %Typical candidates are evaluation on more study objects, investigation of potential threats to validity, … 
 %The point here is to inform the reader (and your supervisor) that you were aware of these limitatons. Limit this chapter to very few pages. 
 %Two is entirely fine, even for a Master’s thesis.
\pagebreak


%------------------------------CONCLUSIONS-----------------------------------
\section{Conclusions}
\label{sec:7}
Short summary of the contribution and its implications. The goal is to drive home the result of your thesis.
Do not repeat all the stuff you have written in other parts of the thesis in detail. Again, limit this chapter to very few pages. 
The shorter, the easier it is to keep consistent with the parts it summarizes.

\pagebreak

%------------------------------REFERENCES-----------------------------------

\addcontentsline{toc}{section}{References}


\bibliographystyle{IEEEtran}
\bibliography{bibi.bib}


\pagebreak

%-----------------------------APPENDIX--------------------------------

\appendix

\section{Appendix 1}




\end{document}
